\begin{abstract}
  This document is the report of a masters study project in the field of hardware programming on an FPGA device. The aim of this project was to gain knowledge in two interesting aspects of FPGA related technologies.\par

  The first aspect is networking on an FPGA device via ethernet and the internet protocol family. A multitude of possible new applications can be enabled, if communication with or control of FPGA devices can be done over ethernet. Some of those possible applications may include light web servers, configuration of FPGA devices over network or logging of tasks to remote computers. Though, at least light web servers would need more network protocols to be implemented.

  The second aspect is programming an FPGA device with the relatively new Vitis HLS workflow. HLS stands for high level synthesis and it uses a higher level language like C++ to compile code down to VHDL or Verilog. This tool can be really a big help in programming more complex systems on an FPGA device, since C++ and several additional libraries provide an enormously helpful level of abstraction, where you don't have to be concerned with implementation details of arithmetic, interfaces and much more.

  The first two sections will provide a quite detailed description of those two aspects. The third section will provide information about the implementation details of the project.
\end{abstract}
